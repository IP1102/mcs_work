Refactoring is a process and a set of techniques to improve source code's structure and quality without impacting the behavior and functionality of the software, thus making the software more maintainable~\cite{Opdyke1992Refactoring, Fowler1999Refactoring}. 
It is a widely used technique among software developers as it improves code readability, testability, flexibility, and adaptability to introduce changes to meet new requirements~\cite{Mens2004Survey}.

One of the critical questions that a developer answers during software development is whether a source code entity (\eg{} a method or a class) requires refactoring and if yes, what is the most appropriate refactoring in the context. 
Practitioners decide \textit{what} to refactor according to their intuition and experience. 
They can also use automated tools to calculate code quality metrics and code smells~\cite{Sharma2018}. 
However, metrics and smells focus on the \textit{problem} aspect and provide little help to decide whether and what refactoring technique must be applied to remove the smell or improve the metrics.


To overcome this challenge, many studies propose techniques to predict refactoring candidates by analyzing source code properties~\cite{Aniche2020Effectiveness, Karakati2023Software, Kurbatova2020Recommendation}.
However, existing efforts in
this direction
% the field of predicting refactoring candidates
suffers from several deficiencies. 
Currently, the state-of-the-art research in detecting refactoring candidates, 
such as Aniche \etal{}~\cite{Aniche2020Effectiveness}, 
follows a metric-based approach, \ie{} it collects code quality and process metrics, and trains a machine learning model using the collected metrics as features.
Similarly, Xu \etal{}~\cite{Xu2017}, use variable accesses in addition to code quality metrics.
These approaches fail to capture the hidden contextual and syntactical characteristics of code that might contribute to better refactoring candidate identification. 
To overcome the issue,
researchers have used code embeddings~\cite{Kurbatova2020Recommendation, Karakati2023Software} extracted from Code2Vec~\cite{Alon2019Code2vec}. 
However, existing studies in this domain do not capture rich contextual and syntactical characteristics of code~\cite{Karmakae2021Modelcoderep};
such information could significantly improve the performance of the identified refactoring candidates.
Secondly, existing studies lack an appropriate mechanism to define and identify negative samples for their dataset in this context.
A code snippet is considered a negative sample if it is not a candidate for the specific refactoring.
Typically, studies use tools such as RefactoringMiner~\cite{Tsantalis:ICSE:2018:RefactoringMiner, Tsantalis:TSE:2020:RefactoringMiner2.0} to identify positive code samples. 
% \todo{Reformatted the sentence and added ref} 
However, to identify negative samples, researchers define unsound rule-based heuristics
resulting in a low-quality noisy dataset~\cite{trautsch2023really}.
Finally, most of the previous research in this field fails to consider the real-world ratio of positive and negative samples while evaluating the predictive models. 
Ignoring this guideline
results in a model that works well in an experimental study but performs poorly when deployed in a real-world context~\cite{Sharma2021, DiNucci2018}.

% We attempt to address the deficiencies mentioned above in this paper.
In this thesis, we address the aforementioned deficiencies.
We present an automated Deep Learning (\dl{})-based technique to identify candidates for \exm{} refactoring. 
 %It specifically focuses on identifying the method level code refactoring \emph{extract method}. 
The \exm{} refactoring isolates a code block from a larger method and generates a new method based on the extracted code snippet~\cite{Fowler1999Refactoring}. 
We kept our focus on \exm{} because it is one of the most commonly used refactoring~\cite{Murphy-Hill2012}. 
We create our dataset from open-source Java repositories and 
prepare an effective code representation capturing syntactic and semantic properties of methods by combining \GCB{}~\cite{Guo2020GraphCodeBERT} and Autoencoder~\cite{Liou2014Autoencoder}.
The representation is then used to identify \exm{} refactoring candidates.

\textit{Contributions of the study:}
% \begin{itemize}
%     \item
    We propose a novel mechanism to properly identify positive and negative samples for \exm{} refactoring.
    The mechanism helps us create a dataset containing $55,430$ positive and negative samples that serves as a benchmark for automated refactoring candidate identification approaches properly.
  To study the effectiveness of the method representation generated using \GCB{} in a binary classification task, we propose an Autoencoder-based approach to identify latent features.


\textbf{Replication package:} We made our code~\cite{ExtractMethodIdentificationRepo} publicly available with our training data~\cite{indranil_palit_2023_8122619} for easier replication and use.


 


 


