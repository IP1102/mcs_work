\noindent
\textbf{Construct validity:} 
Construct validity measures the degree to which tools and metrics actually measure the properties that they are supposed to measure. 
It concerns the appropriateness of observations and inferences based on measurements taken during the study.
The quality of positive and negative samples extracted from RefactoringMiner and PyDriller poses a threat to validity.
Though RefactoringMiner and PyDriller are state-of-the-art tools that are widely used and considered accurate,
to mitigate the threat,
we randomly picked up a subset of the identified negative samples and manually evaluated the quality of the samples.


Our approach uses a significantly smaller dataset (approximately $5\%$) compared to the dataset used in the state-of-the-art approach.
Though the chosen repositories are representative,
it can be considered a threat to validity.
We chose the smaller dataset
because of the computing resources required for the full-size dataset. 
Additionally, in this study, we aimed to explore the feasibility and effectiveness of the proposed approach.
In the future, we aim to repeat the experiment with a larger dataset.

\noindent
\textbf{External validity:}
External validity concerns the generalizability and repeatability of the produced results.
One of the threats to validity in this thesis is that the approach proposed is exclusive to \exm{} refactoring. 
Using our approach for another kind of refactoring is challenging and requires extensive reworking of the approach used. 
For example, \emph{move method} refactoring moves a method to an appropriate class~\cite{Fowler1999Refactoring}. 
We cannot apply the same approach that we adopted to create \exm{} dataset for \emph{move method} refactoring
because we will not have the code to collect in the refactored commit since the method would have moved to another class. 
A similar challenge is expected when considering other refactoring types, such as \emph{pull up method} and \emph{push down method}.
In the future, we would like to address this challenge and propose an effective and generic dataset-creation approach for different refactoring types.
